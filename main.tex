%\documentclass[sigplan,screen]{acmart}
%\documentclass[newfonts=false,format=sigconf,9pt,letterpaper]{acmart}
\documentclass[sigconf,9pt]{acmart}

\usepackage{color}
\usepackage{xspace}
\usepackage{graphicx}
\usepackage{subcaption}
\usepackage[english]{babel}
\usepackage{textcomp}
\usepackage{times}
\usepackage{amsmath}
\usepackage{setspace}
\usepackage[inline]{enumitem}
\usepackage{natbib}
\usepackage{graphicx}
\usepackage[normalem]{ulem}
\usepackage{soul}
\usepackage{latexsym}
\usepackage{bbding}
\usepackage{pbox}

\usepackage{array}
\newcounter{rowcount}
\setcounter{rowcount}{-1}
%MAN: Memory alloc for Networking
%DAMN: Direct Access Memory for Networking
% Segregated Memory N
\newcommand{\oursys}{MAIO\xspace}
\newcommand{\size}{64KB\xspace}
\newcommand{\sockets}{BSD Sockets\xspace}
\newcommand{\X}{\textcolor{red}{\XSolidBrush}}
\newcommand{\V}{\textcolor{green}{\Checkmark}}
\newcommand{\igor}{\textcolor{blue}{IgorY}\xspace}
%%%%%%%%%%%%%%%%%%%%% Commands from prev version
\usepackage[utf8]{inputenc}
\title{Rethinking Zero-Copy Networking with \oursys}

%\author{markuzea Markuze}
%%% potentially enable (i.e. remove the disable command) for the fields below in the camera-ready version          
\settopmatter{printacmref=false}
%\settopmatter{printfolios=false,printacmref=false} %numbers the pages; remove the ugly ACM reference
\setcopyright{none}
\renewcommand\footnotetextcopyrightpermission[1]{} % removes footnote with conference information in first column
\pagestyle{plain} % removes running headers

\begin{document}

\begin{abstract}
    Berkeley Sockets (a.k.a, BSD, POSIX sockets) are ubiquitously used for network communication. \sockets have been the \emph{de-facto} standard API for network I/O since they were introduced almost four decades ago. 
    
    With the advent of high-speed Ethernet, the performance overhead of \sockets became evident. Attempts to avoid these overheads have spurred a trend kernel bypass techniques, e.g., DPDK, AF\_XDP, Netmap. By bypassing the kernel, these methods attempt to avoid the performance penalties associated with \sockets, i.e., memory copy, system calls, and a slow network stack. However, with great performance comes the great responsibility of re-creating the same network infrastructure that already exists inside the kernel. Kernel developers attempt to close the performance gap by adding new capabilities, most notably XDP, MSG\_ZEROCOPY, and tcp\_mmap, but none of the proposed solutions is a panacea.
    
    In this work, we propose a new paradigm for userspace networking, aiming to shrink the performance gap between \sockets and kernel bypass techniques, allowing application developers to keep the standard \sockets API, network stack (e.g., TCP) and network tools without compromising on performance. We introduce \oursys, a dedicated memory allocator for networking that inherently facilitates zero-copy I/O operations. We modify the kernel memory management system to implement dynamic memory segregation, introducing \textit{I/O only pages} that are shared between the user, the kernel, and the device. These I/O pages are used only for I/O and can never be used by the kernel for any other purposes. This scheme facilitates zero-copy I/O while isolating kernel memory from the user. Additionally, we leverage existing HW capabilities~(e.g., NVIDIA QPs, Intel ADQ) to facilitate isolation between processes.


    \oursys is the first design to provide zero-copy networking while still taking advantage of the robust kernel network stack without compromising the system's security. It is currently used to facilitate efficient networking for our companies next-generation SD-WAN gateways.
    
\end{abstract}

\maketitle
\sloppypar
%\section{TODO}
%\begin{enumerate}
%    \item Complete Eval.
%    \item Consider Fig with pps and BW on page 1.
%    \item Complete background and table.
%    \item Merge Intro with background?    
%    \item Adjust fig sides.
%\end{enumerate}
\section{Introduction}
\sockets provide a convenient API for user-space I/O, but since their inception network speeds have outstripped those of the CPU. Furthermore, while CPU and memory clock speeds have stagnated in the past decade, Ethernet speeds are growing steadily ~\cite{roadmap}. The greatest performance penalty is due to memory copying, which can take a big part of the CPU cycles ~\cite{desendmsg} and hurt other processes by "polluting" the shared L3 cache and putting additional pressure on the memory channels \cite{markuze2016true}. 

The performance hit due to memory copying is well known. Attempts to ameliorate the costs of moving data have spawn numerous partial optimizations in the kernel (e.g., splice, sednfile, MSG\_ZEROCOPY, tcp\_mmap). A comprehensive solution is missing resulting in the adoption of kernel bypass technologies (e.g., KTCP, Netmap, XDP). Kernel bypass solutions eschew the rich networking infrastructure, developed and perfected in the kernel, leaving the developers with the need to re-develop existing infrastructure (e.g., IP,TCP,ICMP,).

There is a rich literature base spanning 50 years on zero-copy techniques, non provide a full solution (see Related Work). \oursys is an additional step forward towards a usable zero copy technique, that can be applicable both to virtual and physical environments.
\section{Background}
\subsection{eBPF}
XDP,SOCKMAP,io ring.
\subsection{Kernel Bypass}
Netmap,DPDK.


\section{Design and implementation}\label{sec:design}
Our main goal when designing \oursys was to preserve a standard \sockets API, while providing seamless zero-copy I/O support. A secondary goal is to eliminate costly system calls from the data path. 

We can classify the existing zero copy techniques into four categories:
\begin{enumerate}
    \item Dynamic remapping (e.g., tcp\_mmap, MSG\_ZEROCOPY\cite{desendmsg}, mikelangelo-project\cite{mikelangelo}).
    \item Kernel Bypass (e.g., DPDK, Netmap\cite{rizzo2012netmap}).
    \item Special/Limited use-case (splice, sendfile).
    \item Shared buffers (e.g, INSTANCE \cite{instance}).
\end{enumerate}
Out of the four categories only \#1 and \#4 allow for a generic use of standard \sockets API. Due to mainly security concerns examples that fit into category \#4 are the rarest.
While there are many examples to solutions that fall into category \#1 \cite{mikelangelo-empty,desendmsg}, but they usually discover that modifying virtual memory on the fly is computationally expensive (Sec .\ref{sec:eval}). 

For these reasons we have decided that we should explore a shared buffer solution. Our inspiration for a correct shared buffer implementation comes from a similar problem in IOMMU security, where the cost of dynamic remapping IOMMU address was resulting in poor performance. DAMN \cite{markuze2018damn}, creates a memory allocator over a pool of perpetually DMA mapped pages, which are used exclusively for I/O by the Linux Network stack and device drivers. This solution effectively creates a \emph{secure} shared memory solution between the Server and the NIC.

We propose \oursys, a Memory Allocator for I/O, for the  exclusive use of a specific user-process and the device drievr. The \oursys allocator uses a pool of dedicated compound memory pages(i.e., \_\_GFP\_COMP). We adopt the allocation mechanics proposed in DAMN\cite{markuze2018damn}. I.e., the allocator is based on two known mechanisms; a page\_frag mechanism \cite{pagefrag} over \size buffers, these buffers in turn are allotted by a magazine allocator \cite{bonwick2001magazines}. This allocation scheme allows for efficient allocation of variable size buffers in the kernel. Variable size allocation is needed to support variable sizes of MTU and HW offloads (e.g., HW GRO). To facilitate zero copy, these pages are mapped \emph{once} to the virtual memory address space of the privileged user-space process. In order to use \oursys, the user-space program, has to mmap the \oursys buffer and then allocate a virtual region for its own use (i.e., zero-copy send), the size of the allocated region should be a multiple of \size. A second way the user-space process can get \oursys buffers, is by performing zero-copy receive. The user-space process can return memory to the kernel via a exception-less mechanism described in Sec. \ref{sec:bifurcated}.
The \oursys buffers are depicted in Fig. \ref{fig:our_sys} as dashed boxes, same pages are used in TX and RX by the user (marked \textbf{[a]} and \textbf{[b]}) and TX/RX by the device driver (marked \textbf{[c]} and \textbf{[d]}). 
%That process is now able of perform zero-copy I/O.
\subsection{Bifurcated I/O}\label{sec:bifurcated}
A second issue that impacts user space I/O performance, is the direct and indirect cost of system calls\cite{flexsc}.
To avoid the costly operation of invoking a costly system calls we offload the I/O operation to a dedicated kernel thread (Fig. \ref{fig:our_sys} \textbf{(3)}) which will perform the I/O operation using kernel sockets \cite{ktcp}. For example a \texttt{send\_msg} system call is replaced with an I/O descriptor (i.e., \texttt{struct msghdr} and \texttt{int flags}) written to a shared memory ring buffer (Fig. \ref{fig:our_sys} \textbf{(2)}). This form of splitting the responsibility for performing I/O preservers the existing socket API, facilitates exception-less system calls, and allows for better parallel programming. Bifurcated I/O enables the separation of the application computations and the TCP computations to different CPU cores. Both zero-copy and standard send are supported, in standard mode the sent buffer is copied to a new \oursys buffer before send. Supporting standard mode of send is beneficial for applications performing small I/O e.g., 64B sends (we evaluate the costs in Sec .\ref{sec:eval_bif}).

In additional dedicated kernel threads are used to perform memory operations i.e., retrieving memory buffers back from the user.%RX: just poll the descriptor ring or sleep with sys call.
%TX: continuous poll. NAPI like execution, with amortized sys call, when not polling.

\begin{figure}[t]
    \centering
    \includegraphics[width=0.8\columnwidth]{ktcp_z.pdf}
    \caption{1. \oursys shared memory buffers 2. shared io rings for exception-less system calls 3. A kernel thread executing I/O operations 4. A dedicated RX ring.
    [a] zero-copy send\_msg [b] zero-copy recvmsg [c] \oursys buffer in driver TX ring [d] \oursys buffers used by the device driver for RX}
    \label{fig:our_sys}
\end{figure} 

\subsection{Security}
Such a solution initially razes concerns about the security and stability of the system, as the process now \emph{seemingly} has access to sensitive kernel memory. 

\noindent\textbf{Driver Support.} We can allocate dedicated RX rings on the NIC for \oursys users. HW support\cite{flow_direct} can direct a single 5 tuple (or a defined group of 5 tuples). Limiting the shared buffers \emph{only} to this users data. The dedicated RX ring shown in Fig. \ref{fig:our_sys} \textbf{(4)} is using \oursys pages (Fig. \ref{fig:our_sys} \textbf{[d]}). In Fig. \ref{fig:our_sys}, we also see the kernel buffer (marked as the opaque box) being used by the same TX ring as a \oursys page(Fig. \ref{fig:our_sys} \textbf{[c]}), no special care is needed for data separation. The implementation of a driver support, is out of the scope of this work. 

\noindent\textbf{Kernel Security.} \oursys is integrated in such a way that the shared pages are only ever used by the Kernel to hold the I/O \emph{data} buffers and \emph{not} the meta data or any other kernel need. Namely, the process can only ever see the information it has written or data bound to user-space. In addition to the data, the process is privy to the transport headers as well; we assume the NIC supports Header/Data splitting\cite{hds} which can place the headers onto non-shared buffers.

\noindent\textbf{User Security.} By sharing all potential RX buffers \oursys exposes all traffic to a single observer.
Without driver support this limits the usefulness of \oursys to those cases when the user is trusted (e.g., sudo). 

\subsection{Shared buffer concerns.}
\noindent\textbf{Kernel Starvation.} The user process may hoard \oursys buffers without releasing them to the kernel.
In this case the driver will revert to standard memory allocation, and will render the application unable to receive, while other process and kernel functionality will remain intact.


\noindent\textbf{Pinned pages.} Zero-copy solutions with shared static buffer, were once considered dangerous because these shared pages can be exhausted and cannot be swapped out \cite{song2012performance,yamagiwa2005active}. We contend that this is not a real concern for modern systems as systems with hundreds of GB are the norm. Key/Value applications (e.g., memcached, redis) expect their memory to be persistent in memory. Additionally HPC applications, many\cite{top500} of which use RDMA, \texttt{register} (i.e., pin to memory) large memory regions that are then used for I/O.

\subsection{zero copy support for kernel sockets}
We expand the existing Linux TCP API with a \texttt{tcp\_read\_sock\_zcopy} for \texttt{RX} and add a new msg flag \texttt{SOCK\_KERN\_ZEROCOPY} for \texttt{tcp\_sendmsg\_locked} in \texttt{TX}. 
\noindent \textbf{RX.} We base our new function \texttt{tcp\_read\_sock\_zcopy} on existing infrastructure i.e., \texttt{tcp\_read\_sock}. It is used by \texttt{tcp\_splice\_read} to collect buffers from a socket without copying.

\noindent \textbf{TX.} Zero-copy infrastructure already exists in the form of MSG\_ZEROCOPY\cite{desendmsg}. When kernel memory is used for I/O, enabling zero copy is trivial when compared to zero copy from user space. The pages are already pinned in memory and there is not need for a notification on \texttt{TX} completion. The pages are reference counted, and can be freed by the device driver completion handler.


\noindent \textbf{\texttt{do\_tcp\_sendpages}.}
Instead of modifying the behavior of \texttt{tcp\_sendmsg\_locked}, its also possible to use \texttt{do\_tcp\_sendpages}, which is used in splice. Ironically, \texttt{do\_tcp\_sendpages} accepts only one page fragment (i.e., \texttt{struct page}, size and offset) per invocation and does not work with a scatter-gather list, which \texttt{tcp\_sendmsg\_locked} supports.

%Related to FlexSC \cite{flexsc}, \\TODO:\\
%1. ~/memory\_trace/poller/\\
%2. rerun context switch tests?
\section{Evaluation}
We evaluate the benefits of \oursys on a virtual environment in the Google Clod Platform(GCP).
For our testing we have crated three VMs; 16 core\footnote{These VMs have two threads per physical core} Intel Cascade Lake high-throughput VMs capable of 32Gb/s of  egress bandwidth\cite{gcp}, each with 64GB of RAM.

\subsection{Bifurcated TX}\label{sec:eval_bif}
We evaluate the direct benefit of bifurcated system calls; with a UDP stream of 64 Byte packets.
We test a simple \texttt{send\_msg}, a kernel thread performing \texttt{kernel\_send\_msg} and \oursys.

\subsection{Zero Copy TX}
We evaluate the cost of data copying for single core and multi core sends. To minimise the impact of system calls on performance we send 256KB buffers. We evaluate a simple \texttt{send\_msg}, a \texttt{send\_msg} with \texttt{MSG\_ZERO\_COPY}, \textcolor{red}{\texttt{sendfile}} and \oursys.

\begin{figure}[t]
    \centering
    \includegraphics[width=\columnwidth]{Figure_1.pdf}
    \caption{The average cycles per packet need for each splicing solution; taken over 10 iterations runs }
    \label{fig:cyc_byte}
\end{figure}
\begin{figure}[t]
    \centering
    \includegraphics[width=\columnwidth]{bifurcated.pdf}
    \caption{Number of 64B udp packets sent using user-space sockets, kernel sockets and MAIO. \textcolor{red}{consider mitigations off...}}
    \label{fig:pps}
\end{figure}
\begin{figure}[t]
    \centering
    \includegraphics[width=\columnwidth]{bifurcated.pdf}
    \caption{\textcolor{red}{Placeholder}}
    \label{fig:tx_compare}
\end{figure}
\subsection{Socket Splicing}
We perform a simple TCP socket splicing experiment (i.e., Mvoing bytes from one TCP socket to another) akin to the TCP-split functionality of KTCP. We measure the effectivness of each proposed solution by looking the the number of CPU cycles spent on average to splice (i.e., receive and send) a single byte.

We use 256KB buffers both for RX and TX, rendering the effect of system calls negligible. System calls account for less than 1\% of the cpu syscalls of iosubmit,splice and naive. We use perf\cite{perf} to analyze the major contributors to the cycles per byte costs of each of tested techniques. Unsurprisingly, KTCP, iosubmit and naive all spend between 25\% to 32\% of the cycles on memory copying. We can see the results in Fig. \ref{fig:cyc_byte}.

%Why we are \emph{not} touching DPDK and friends with a stick.

%Splice Compare (Check Poll):
%\begin{itemize}
%    \item Naive
%    \item SOCKMAP
%    \item splice
%    \item vmsplice  (?)
%    \item io\_remap (?)
%    \item ktcp - kernel TCP client + halfduplex.
%    \item \oursys - kernel zero TCP client + read/write op. 
%    \item ktcp\_zero. - kernel zero TCP client
%\end{itemize}


%\subsection{BW - cycles/byte}
%\subsection{Latency TCP/RR}
%\subsection{Scale - multiple connections - BW,Latency}

\section{Related Work}
Review of zcopy techniques:\\
\url{http://www.cscjournals.org/manuscript/Journals/IJCSS/Volume6/Issue4/IJCSS-756.pdf}\\
Lyranet:\\
\url{https://webpages.uncc.edu/~jmconrad/EmbeddedSystems/TCP_IP\%20protocol\%20stack.pdf}\\
Instance:\\
\url{https://heim.ifi.uio.no/paalh/instance/espen.pdf}
ZeroCopy Paravirt:\\
\url{https://events19.linuxfoundation.org/wp-content/uploads/2017/12/Empty-Promise-Zero-Copy-Receive-for-vhost-Mike-Rapoport-IBM.pdf}

\section{Conclusion}
In this work, we have presented \oursys, a novel paradigm for user-space networking. \oursys facilitates zero-overhead network I/O, i.e., zero-copy and exception-less system calls. Unlike previous solutions, \oursys preservers the ubiquitous \sockets API without sacrificing the system's performance or safety.
%There is a rich literature base spanning 50 years on zero-copy techniques, non provide a full solution\cite{song2012performance}. 
\oursys is a much-needed step forward towards a performant and generic zero-copy technique. %A solution that we presume can be applied both to virtual and physical environments.
%We Rule.
%``I always thought something was fundamentally wrong with the universe'' \citep{adams1995hitchhiker}

\bibliographystyle{plain}
\bibliography{references}
\end{document}
