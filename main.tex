%\documentclass[sigplan,screen]{acmart}
\documentclass[newfonts=false,format=sigconf,9pt,letterpaper]{acmart}

\usepackage{color}
\usepackage{xspace}
\usepackage{graphicx}
\usepackage{subcaption}
\usepackage[english]{babel}
\usepackage{textcomp}
\usepackage{times}
\usepackage{amsmath}
\usepackage{setspace}
\usepackage[inline]{enumitem}
\usepackage{natbib}
\usepackage{graphicx}
\usepackage[normalem]{ulem}
\usepackage{soul}

\newcommand{\oursys}{MAIO\xspace}
\newcommand{\sockets}{BSD Socekts\xspace}
\newcommand{\igor}{\textcolor{blue}{IgorY}\xspace}
%%%%%%%%%%%%%%%%%%%%% Commands from prev version
\usepackage[utf8]{inputenc}
\title{Rethinking User Space I/O with \oursys}

%\author{markuzea Markuze}
%%% potentially enable (i.e. remove the disable command) for the fields below in the camera-ready version          
\settopmatter{printacmref=false}
%\settopmatter{printfolios=false,printacmref=false} %numbers the pages; remove the ugly ACM reference
\setcopyright{none}
\renewcommand\footnotetextcopyrightpermission[1]{} % removes footnote with conference information in first column
\pagestyle{plain} % removes running headers

\begin{document}

\begin{abstract}
    Berkeley Sockets (a.k.a, BSD,POSIX sockets) are ubiquitously used for inter process communication. \sockets have been the \emph{de-facto} standard API for I/O since they were introduced almost four decades ago. 
    
    With the advent of high-speed Ethernet, the performance costs of \sockets became evident. These performance costs are particularly burdensome on HTTP proxies and load balancers. The main function of which is to splice TCP sockets, performing two I/O operations per I/O buffer, namely a read and a write. The performance costs of user-space I/O, has launched a trend of kernel bypass technologies (e.g, DPDK,Netmap). By bypassing the kernel, these methods attempt to avoid the performance penalties associated with \sockets; memory copy, system calls and a slow network stack. But, with great performance comes the responsibility of re-creating the same network infrastructure that already exists inside the kernel. Kernel developers tried to close the performance gap by adding new capabilities, most notably XDP, MSG\_ZEROCOPY and tcp\_mmap, but none of the proposed solutions is a panacea.
    
    In this work we propose a new paradigm for user space I/O aiming to shrink the performance gap, we introduce \oursys (pronounced \emph{Mayo}). \oursys is a dedicated memory allocator for I/O operations that inherently facilitates efficient zero-copy I/O operations. We evaluate a prototype of \oursys, for the use case of TCP socket splicing comparing to  the state of the art methods (e.g, SOCKMAP, KTCP, io\_submit). We find that \oursys is a \emph{boon for TCP splicing}. 
    
    \oursys is potentially beneficial for many VMware networking products like NSX and project Pathway. The new paradigm is also applicable to ESX guest networking, potentially diminishing the performance gap between SR-IOV and VMXNET3. 
    
    %The problem of splicing sockets is not new but no adequate solutions exist. Splicing sockets present three performance problems; context switch costs, system call costs and memory copying. We present \oursys, a modified Linux kernel tailored for I/O applications. We extensively evaluate the existing state of the art solutions (e.g, SOCKMAP, KTCP) and show \emph{magical} improvements in both I/O latency and the bytes/cycle metrics. We farther explore how \oursys can be used to enhance user space I/O by implementing and evaluating seamless (w/o system calls) send/receive calls for small I/O.
 %   With the advent of multi GbE networking userspace I/O overheads take more and more of system resources. In this work we present \oursys which provides async I/O and zero copy networking.
    
\end{abstract}

\maketitle
\sloppypar

\section{Introduction}
\begin{itemize}
    \item Bifurcated I/O. Related work: FlexSC,NetMap(?)
    \item Shared Memory Alloc network I/O.\\ New, Instead of playing with user page tables share mem.\\ Related work: LyraNet, MSG\_ZEROCOPY\cite{desendmsg}
\end{itemize}
\smallskip
TODO:
\begin{enumerate}
    \item Export to uspace - (Follow Erics patch...) or use READ/Write.
    \item kernel client for splice (plus tester).
    \item data collection scripts.
    \item prep Env for testing (For \igor).
    \item Test Client/Server (\igor)
    \item \st{Zcopy splice - Fix bug}.
    \item \st{DAMN on PWY kernel}.
    \item \st{Pre-load MANE (Mem Alloc for NEtwork) to user.} 
    \item \st{Bifurcated Send + (descriptors).}
\end{enumerate}
\smallskip


\section{Background}
L7 splicing and state of the art.
\section{\oursys}
\subsection{zero copy support for kernel sockets}
\subsection{Shared Network Memory allocator}
\subsection{Security}
\subsection{Seamless/bifurcated I/O}
\section{Evaluation}
Why we are \emph{not} touching DPDK and friends with a stick.

Splice Compare (Check Poll):
\begin{itemize}
    \item Naive
    \item SOCKMAP
    \item splice
    \item vmsplice  (?)
    \item io\_remap (?)
    \item ktcp - kernel TCP client + halfduplex.
    \item \oursys - kernel zero TCP client + read/write op. 
    \item ktcp\_zero. - kernel zero TCP client
\end{itemize}
\smallskip
Stream Test:
\begin{itemize}
    \item MSG\_ZEROCOPY
    \item tcp\_mmap
    \item \oursys
    \item kernel zero TCP client.
\end{itemize}
\smallskip
To Collect:
\begin{itemize}
    \item CPU (exists)
    \item B/W, packets (parse proc file)
    \item perf record (functions - just do it.)\\
    Show cycles per packet.
    \item memory usage.
\end{itemize}
\subsection{BW - cycles/byte}
\subsection{Latency TCP/RR}
\subsection{Scale - multiple connections - BW,Latency}
\section{Conclusion}
We Rule.

``I always thought something was fundamentally wrong with the universe'' \citep{adams1995hitchhiker}

\bibliographystyle{plain}
\bibliography{references}
\end{document}
