%\documentclass[sigplan,screen]{acmart}
%\documentclass[newfonts=false,format=sigconf,9pt,letterpaper]{acmart}
\documentclass[sigconf,9pt]{acmart}

\usepackage{color}
\usepackage{xspace}
\usepackage{graphicx}
\usepackage{subcaption}
\usepackage[english]{babel}
\usepackage{textcomp}
\usepackage{times}
\usepackage{amsmath}
\usepackage{setspace}
\usepackage[inline]{enumitem}
\usepackage{natbib}
\usepackage{graphicx}
\usepackage[normalem]{ulem}
\usepackage{soul}

\newcommand{\oursys}{MAIO\xspace}
\newcommand{\sockets}{BSD Sockets\xspace}
\newcommand{\igor}{\textcolor{blue}{IgorY}\xspace}
%%%%%%%%%%%%%%%%%%%%% Commands from prev version
\usepackage[utf8]{inputenc}
\title{Rethinking User Space I/O with \oursys}

%\author{markuzea Markuze}
%%% potentially enable (i.e. remove the disable command) for the fields below in the camera-ready version          
\settopmatter{printacmref=false}
%\settopmatter{printfolios=false,printacmref=false} %numbers the pages; remove the ugly ACM reference
\setcopyright{none}
\renewcommand\footnotetextcopyrightpermission[1]{} % removes footnote with conference information in first column
\pagestyle{plain} % removes running headers

\begin{document}

\begin{abstract}
    Berkeley Sockets (a.k.a, BSD,POSIX sockets) are ubiquitously used for inter process communication. \sockets have been the \emph{de-facto} standard API for I/O since they were introduced almost four decades ago. 
    
    With the advent of high-speed Ethernet, the performance costs of \sockets became evident. These performance costs are particularly burdensome on HTTP proxies and load balancers. The main function of which is to splice TCP sockets, performing two I/O operations per I/O buffer, namely a read and a write. The performance costs of user-space I/O, has launched a trend of kernel bypass technologies (e.g, DPDK,Netmap). By bypassing the kernel, these methods attempt to avoid the performance penalties associated with \sockets; memory copy, system calls and a slow network stack. But, with great performance comes the responsibility of re-creating the same network infrastructure that already exists inside the kernel. Kernel developers tried to close the performance gap by adding new capabilities, most notably XDP, MSG\_ZEROCOPY and tcp\_mmap, but none of the proposed solutions is a panacea.
    
    In this work we propose a new paradigm for user space I/O aiming to shrink the performance gap, we introduce \oursys (pronounced \emph{Mayo}). \oursys is a dedicated memory allocator for I/O operations that inherently facilitates efficient zero-copy I/O operations. We evaluate a prototype of \oursys, for the use case of TCP socket splicing comparing to  the state of the art methods (e.g, SOCKMAP, KTCP, io\_submit). We find that \oursys is a \emph{boon for TCP splicing}. 
    
    \oursys is potentially beneficial for many VMware networking products like NSX and project Pathway. The new paradigm is also applicable to ESX guest networking, potentially diminishing the performance gap between SR-IOV and VMXNET3. 
    
    %The problem of splicing sockets is not new but no adequate solutions exist. Splicing sockets present three performance problems; context switch costs, system call costs and memory copying. We present \oursys, a modified Linux kernel tailored for I/O applications. We extensively evaluate the existing state of the art solutions (e.g, SOCKMAP, KTCP) and show \emph{magical} improvements in both I/O latency and the bytes/cycle metrics. We farther explore how \oursys can be used to enhance user space I/O by implementing and evaluating seamless (w/o system calls) send/receive calls for small I/O.
 %   With the advent of multi GbE networking userspace I/O overheads take more and more of system resources. In this work we present \oursys which provides async I/O and zero copy networking.
    
\end{abstract}

\maketitle
\sloppypar

\section{TODO}
\begin{itemize}
    \item Bifurcated I/O. Related work: FlexSC,NetMap(?)
    \item Shared Memory Alloc network I/O.\\ New, Instead of playing with user page tables share mem.\\ Related work: LyraNet, MSG\_ZEROCOPY\cite{desendmsg}
\end{itemize}
\smallskip
TODO:
\begin{enumerate}
    \item Start experiments (\igor).
    \item Export to uspace - (Follow Erics patch...) or use READ/Write - \oursys.
    \item \st{Data collection scripts.}
    \item \st{kernel client for splice (plus tester).}
    \item \st{Test Client/Server }(\igor).
    \item \st{Test description table.}
    \item \st{Zcopy splice - Fix bug}.
    \item \st{DAMN on PWY kernel}.
    \item \st{prep Env for testing }(For \igor).
    \item \st{Pre-load MANE (Mem Alloc for NEtwork) to user.} 
    \item \st{Bifurcated Send + (descriptors).}
\end{enumerate}

\section{Introduction}
\sockets provide a convenient API for user-space I/O, but since their inception network speeds have outstripped those of the CPU. Furthermore, while CPU and memory clock speeds have stagnated in the past decade, Ethernet speeds are growing steadily ~\cite{roadmap}. The greatest performance penalty is due to memory copying, which can take a big part of the CPU cycles ~\cite{desendmsg} and hurt other processes by "polluting" the shared L3 cache and putting additional pressure on the memory channels \cite{markuze2016true}. 

The performance hit due to memory copying is well known. Attempts to ameliorate the costs of moving data have spawn numerous partial optimizations in the kernel (e.g., splice, sednfile, MSG\_ZEROCOPY, tcp\_mmap). A comprehensive solution is missing resulting in the adoption of kernel bypass technologies (e.g., KTCP, Netmap, XDP). Kernel bypass solutions eschew the rich networking infrastructure, developed and perfected in the kernel, leaving the developers with the need to re-develop existing infrastructure (e.g., IP,TCP,ICMP,).

There is a rich literature base spanning 50 years on zero-copy techniques, non provide a full solution (see Related Work). \oursys is an additional step forward towards a usable zero copy technique, that can be applicable both to virtual and physical environments.

\section{Background}
The quest for zero-copy techniques has yielded multiple solutions for zero copy I/O, we will discuss the solutions that are used today (i.e., passed the test of time) and those they may seem related to \oursys. Several previous works have made reviews on zero-copy and fast packet processing techniques\cite{song2012performance,tsiamoura2014survey}. 
\begin{table*}[]
    \centering
    \begin{tabular}{@{\stepcounter{rowcount}\therowcount.)\hspace*{\tabcolsep}}l|c|c|c|c|c|c|l}\hline
        System  & Copy & \pbox{2cm}{System\\Call} & Zero Overhead & \pbox{2cm}{Static\\mapping} & \pbox{2cm}{Network\\ Stack} &  generic use & comments\\\hline
         Naive & 1 & 1 & \X & \V & \V & \V & \\ 
         splice & 0 & 1 & \X & \V & \V & \X & Pipe needed in Linux\\ 
         sendfile & 0 & 1 & \X & \V & \V & \X & Send File only\\ 
         vmsplice & 0 & 1 & \X & \X & \V & \X & No completion notification\\
         SOCKMAP & 0 & 0 & \X & \V & \V & \X & Splicing Only, eBPF\\ 
         io\_map & 0 & 1 & \X & \V & \V & \X & \\ 
         NetMap \cite{rizzo2012netmap} & 0  & 0 & \V & \V & \X & \V &\\
         DPDK \cite{dpdk}& 0 & 0 & \V & \V & \X & \V &\\
         MSG\_ZEROCOPY & 0 & 1 & \X & \X & \V & \V &\\
         tcp\_mmap & 0 & 1 & \X & \X & \V & \X & Full Page size receive\\
         LyraNet & 0 & 1 & \X & \X & \V & \X & \textcolor{red}{\textbf{Please fix wrong lines...}}\\
         INSTANCE & 0 & 1 & \X & \X & \V & \X & Fixed size buffers\\\hline
         RDMA & 0 & 1 & \V & \V & RDMA & RDMA & Specialized HW\\\hline
         \oursys & 0 & 0* & \V & \V & \V & \V &\\\hline
    \end{tabular}
    \caption{Existing Host I/O solutions}
    \label{tab:sol_compare}
\end{table*}

\subsection{eBPF}
XDP,SOCKMAP,io ring.
\subsection{Kernel Bypass}
Netmap,DPDK.
Presumably, when using RAW sockets, \oursys, should behave similarly to NetMap i.e., a shared memory buffer sent/received directly to/from a dedicated TX/RX ring.
NetMap has non standard API, and can never use the Network Stack. Not sure about the message sizes I assume they are fixed or limited in size in NetMap...
We don't have any of these limitations.

\subsection{Socket Splicing - background}
Socket splicing is major area of interest with multiple projects performing HTTP proxy services( \cite{squid,HAProxy,varnish,nginx,ktcp}). To note, NGINX\cite{nginx} and KTCP\cite{ktcp} are used in VMware products.




\section{Evaluation}
Why we are \emph{not} touching DPDK and friends with a stick.

Splice Compare (Check Poll):
\begin{itemize}
    \item Naive
    \item SOCKMAP
    \item splice
    \item vmsplice  (?)
    \item io\_remap (?)
    \item ktcp - kernel TCP client + halfduplex.
    \item \oursys - kernel zero TCP client + read/write op. 
    \item ktcp\_zero. - kernel zero TCP client
\end{itemize}


\subsection{BW - cycles/byte}
\subsection{Latency TCP/RR}
\subsection{Scale - multiple connections - BW,Latency}

\section{Related Work}
Review of zcopy techniques:\\
\url{http://www.cscjournals.org/manuscript/Journals/IJCSS/Volume6/Issue4/IJCSS-756.pdf}\\
Lyranet:\\
\url{https://webpages.uncc.edu/~jmconrad/EmbeddedSystems/TCP_IP\%20protocol\%20stack.pdf}\\
Instance:\\
\url{https://heim.ifi.uio.no/paalh/instance/espen.pdf}\\
ZeroCopy Paravirt:\\
\url{https://events19.linuxfoundation.org/wp-content/uploads/2017/12/Empty-Promise-Zero-Copy-Receive-for-vhost-Mike-Rapoport-IBM.pdf}

\section{Future Work}
We plan on testing a full version of \oursys kernel including a dedicated RX buffer support on a physical setup. With that implementation, we plan on testing the performance benefit for real user applications beyond socket splicing (e.g., Memcached).

Also, we are keen on exploring the potential benefits of \oursys a para-virtual system. We believe, that a perpetual exposure of dedicated host memory can be a boon to para-virtual networking. Potentially allowing for full zero-copy I/O from the user-space of a guest OS.
\section{Conclusion}
We Rule.

``I always thought something was fundamentally wrong with the universe'' \citep{adams1995hitchhiker}

\bibliographystyle{plain}
\bibliography{references}
\end{document}
