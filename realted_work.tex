%\section{Related Work}
%Review of zcopy techniques:\\
%\url{http://www.cscjournals.org/manuscript/Journals/IJCSS/Volume6/Issue4/IJCSS-756.pdf}\\
%Lyranet:\\
%\url{https://webpages.uncc.edu/~jmconrad/EmbeddedSystems/TCP_IP\%20protocol\%20stack.pdf}\\
%Instance:\\
%\url{https://heim.ifi.uio.no/paalh/instance/espen.pdf}\\
%ZeroCopy Paravirt:\\
%\url{https://events19.linuxfoundation.org/wp-content/uploads/2017/12/Empty-Promise-Zero-Copy-Receive-for-vhost-Mike-Rapoport-IBM.pdf}

\section{Future Work}
%We plan on testing a full version of \oursys kernel including a dedicated RX buffer support on a physical setup. With that implementation, we plan on testing the performance benefit for real user applications beyond socket splicing (e.g., Memcached).

%\subsection{Virtual Environment}
%TODO: some papers also exist on kvm by the same people of "mikelangelo" \cite{mikelangelo}. Need to look at VMXNET3 code... Lots of Lit exists must explore.

We plan on exploring the potential benefits of \oursys for a para-virtual system. We contend that a dedicated segment of host memory can be used for application dedicated para-virtual networking. Potentially allowing for an overhead free(e.g., copy, dynamic virtual memory remapping) zero-copy I/O from a guest OS's user-space, i.e., a \oursys page sent from a guest OS user program used directly by the physical NIC; potentially closing the performance gap between SRIOV and para-virtual devices.
%The improvement of a para-virtual network stack is the subject of current research efforts by other groups as well\cite{mikelangelo, mikelangelo-empty}.



