\section{Introduction}
\sockets provide a convenient API for user-space I/O, but since their inception network speeds have outstripped those of the CPU and memory. Furthermore, while CPU and memory clock speeds have stagnated in the past decade, Ethernet speeds are growing steadily ~\cite{roadmap}. Modern system already need more than just a single CPU core to simply saturate a network link. The greatest performance cost is due to memory copying, which can take a big part of the CPU cycles ~\cite{desendmsg} and hurt other processes by "polluting" the shared L3 cache and putting additional pressure on the memory channels \cite{markuze2016true}. 

The performance hit due to memory copying is well known. Attempts to ameliorate the costs of moving data have spawn numerous partial optimizations in the kernel (e.g., splice, sednfile, MSG\_ZEROCOPY, tcp\_mmap). A comprehensive solution is still missing, resulting in the adoption of kernel bypass technologies (e.g., KTCP, Netmap, XDP). Kernel bypass solutions eschew the rich networking infrastructure, developed and perfected in the kernel, leaving the developers with the need to re-develop existing infrastructure (e.g., IP,TCP,ICMP,IGMP).

In this work we add our solution to the myriad of zero-copy solutions proposed in the past 40 years, none providing all the needed elements .i.e, a simple and generic API coupled with low overheads. We contend that the needed solution has been overlooked due to undue concerns over security and performance. Concerns that are no longer relevant with multi-queue NICs and abundant memory. All that is missing to support \sockets API in the 100+GbE era\cite{roadmap} is to correctly use the existing HW capabilities (i.e., RSS, Multi-Queue NICs). We classify the existing solutions into four main categories(Sec .\ref{sec:design}) and discover that the statically mapped shared buffer approach has remained relatively unexplored in recent years. With most recent techniques (e.g., \cite{xdp,dpdk,rizzo2012netmap,mikelangelo,desendmsg}) focusing instead on dynamic remapping and kernel bypass technologies.

We draw inspiration from previous works on IOMMU security \cite{markuze2016true,markuze2018damn}. In that case, dynamic remapping was incompatible with performance, and yet extensively used. The secure and performant solution was a carefully managed statically mapped memory pool. We show in this work, how this approach can also be used to provide generic zero-copy support for \sockets. 

To provide a fully overhead free I/O, we also work to eliminate the cost of system calls. In this case we adopt the kernel thread approach proposed in FlexSC\cite{flexsc}. We show that shared memory buffers coupled with kernel threads provide a holistic solution for performant \sockets.
We call our system \oursys, a dedicated Memory Allocator for I/O. 

We start by detailing and classifying existing solutions in Sec. \ref{sec:background}. In Sec .\ref{sec:design} we outline the design of \oursys in fine detail and in section \ref{sec:eval} we provide experimental evidence to the benefits of our solution. 

\smallskip
\oursys is potentially beneficial for many VMware networking products like NSX and project Pathway\cite{cbn,ktcp}. Our approach is also applicable to ESX guest networking, potentially diminishing the performance gap between SR-IOV and VMXNET3.